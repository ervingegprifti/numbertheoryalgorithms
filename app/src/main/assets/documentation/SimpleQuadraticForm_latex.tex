\documentclass[12pt,oneside,a4paper]{article}
\usepackage[latin1]{inputenc}
\usepackage{amsmath}
\usepackage{amsfonts}
\usepackage{amssymb}
\usepackage{graphicx}
\usepackage{verbatim}
\usepackage{stmaryrd}
\usepackage{tikz}
\usepackage{xcolor}
\definecolor{bgcolor}{HTML}{E3E1D7}
%https://ftp.cc.uoc.gr/mirrors/CTAN/macros/latex/contrib/algorithm2e/doc/algorithm2e.pdf
\usepackage[linesnumbered,ruled,vlined]{algorithm2e}
% \usepackage{calc} % you're allowed to say \begin{minipage}{\linewidth-2ex}
%\usetikzlibrary{positioning,fit,shapes}
\usepackage[left=20mm, right=20mm, top=20mm, bottom=20mm]{geometry}
\usepackage{url} 
%\usepackage[hidelinks]{hyperref}  % For url links
\usepackage[colorlinks = true, %Colours links instead of ugly boxes
urlcolor = blue, %Colour for external hyperlinks blue, red, black
linkcolor = black, %Colour of internal links
citecolor = red %Colour of citations
]{hyperref}
% [linesnumbered, ruled, vlined]


% Declare variables
\newcommand{\myScale}{0.345}
\newcommand{\myBorder}{0.0pt}
\newcommand{\minipageWidth}{0.485}
\newcommand{\minipageSpaceCenter}{\hspace{5mm}}


\title{Number Theory Algorithms}
\author{Ervin Gegprifti \\ [6pt]
	gegprifti.ervin@gmail.com}
\date{} 

\begin{document}
\pagecolor{bgcolor}	
%\nopagecolor % Transparent if the rendering machine supports it.

\maketitle

\begin{abstract}
	This paper is the documentation for the Simple Quadratic Form module in \href{https://play.google.com/store/apps/details?id=com.gegprifti.android.numbertheoryalgorithms}{Number Theory Algorithms} mobile application.
\end{abstract}

\section*{Simple Quadratic Form}
\begin{equation} \label{eq:SQF}
	bxy+dx+ey=f
\end{equation}	
The equation \eqref{eq:SQF} is a simpler case of the more generic Quadratic Form $ax^2+bxy+cy^2+dx+ey=f$ where $a=c=0$. The algorithm make use of the factoring technique known as (Simon's Favorite Factoring Trick SFFT) in order to find solutions to \eqref{eq:SQF} if any. The implementation of this algorithm is based on (\cite{yan2002number} pg. 56). \\




\begin{algorithm}[H]
	\caption{Simple Quadratic Form Algorithm}
	\DontPrintSemicolon
	\SetAlgoLined
	\KwIn{$b, d,e,f,x,y \in \mathbb{Z}$ and $b \neq 0$}
	\KwOut{$x,y$ solutions if any}
	
	\BlankLine

	Multiply both sides with $b$ then $b^2xy+bdx+bey=bf$ \\
	Add $de$ to both sides then $b^2xy+bdx+bey+de=bf+de$ \\
	The LHS can be written as $(bx+e)(by+d)$ \\
	Let $n=bf+de$ then we must solve $(bx+e)(by+d)=n$ \\
	Factor $n$ into $pq$ pairs \\
	\lIf{there are no $pq$ factors of $n$}{\Return{there is no solution other than the trivial}}
	$solutions$ $\leftarrow$ empty \\
	\For {each $pq$ pair factor of $n$} {
		\eIf{$b \mid (p-e)$ {\bf and} $b \mid (q-d)$}
		{	
			$x=(p-e)/b$ \\ 
			$y=(q-d)/b$ \\
			solutions $\leftarrow$ x,y
		}
		{
			% This is a comment.
			\tcp{there is no integer solution for this pair}
		}
	}

	\Return{solutions}
\end{algorithm}




\begin{thebibliography}{}
	\bibitem{yan2002number}
	Yan, Song Y.
	\newblock {\em Number theory for computing.}
	\newblock {Springer Science \& Business Media}, 2002.
	%\newblock
	%\href{http://...}{Web source...}
\end{thebibliography}
% Source tutorial: https://www.youtube.com/watch?v=NRXvqthfDG0
% Use this link: https://scholar.google.com to find cites for references
% After using \cite command or adding new entry into the bib file. 
% Refer to (\cite{XXX} pg. XX)
% 1. Run pdflatex (F6)
% 2. Run BibTex (F11)
% 3. Run pdflatex (F6)
% 4. Run pdflatex (F6)
%\bibliographystyle{plain} % unsrt % 
%\bibliography{Bibliography}
% Display all bibliography cites even if there are no cites. Or just the cpecific ones
%\nocite{*} 



\end{document}