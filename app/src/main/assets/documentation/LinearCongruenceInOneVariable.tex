\documentclass[12pt,oneside,a4paper]{article}
\usepackage[latin1]{inputenc}
\usepackage{amsmath}
\usepackage{amsfonts}
\usepackage{amssymb}
\usepackage{graphicx}
\usepackage{verbatim}
\usepackage{stmaryrd}
\usepackage{tikz}
% \usepackage{calc} % you're allowed to say \begin{minipage}{\linewidth-2ex}
%\usetikzlibrary{positioning,fit,shapes}
\usepackage[left=20mm, right=20mm, top=20mm, bottom=20mm]{geometry}
\usepackage{url} 
%\usepackage[hidelinks]{hyperref}  % For url links
\usepackage[colorlinks = true, %Colours links instead of ugly boxes
urlcolor = blue, %Colour for external hyperlinks blue, red, black
linkcolor = black, %Colour of internal links
citecolor = red %Colour of citations
]{hyperref}

%https://ftp.cc.uoc.gr/mirrors/CTAN/macros/latex/contrib/algorithm2e/doc/algorithm2e.pdf
\usepackage[linesnumbered,ruled,vlined]{algorithm2e}

% [linesnumbered, ruled, vlined]
\usepackage{xcolor}
\definecolor{bgcolor}{HTML}{E3E1D7}


% Declare variables
\newcommand{\myScale}{0.345}
\newcommand{\myBorder}{0.0pt}
\newcommand{\minipageWidth}{0.485}
\newcommand{\minipageSpaceCenter}{\hspace{5mm}}


\title{Number Theory Algorithms}
\author{Ervin Gegprifti \\ [6pt]
	gegprifti.ervin@gmail.com}
\date{} 

\begin{document}
\pagecolor{bgcolor}
% \nopagecolor % Transparent if the rendering machine supports it.

\maketitle

\begin{abstract}
	This paper is the documentation for the Linear Congruence In One Variable module in \href{https://play.google.com/store/apps/details?id=com.gegprifti.android.numbertheoryalgorithms}{Number Theory Algorithms} mobile application.
\end{abstract}

\section*{Linear Congruence In One Variable}	
The Linear Congruence In One Variable $ax \equiv b \pmod{m}$ is equivalent to the Linear Diophantine Equation In Two Variables $ax-my=b$. If $GCD(a,m) \nmid c$ there is no solution modulo $m$ and if $GCD(a,m) \mid c$ there are $g$ incongruent solutions modulo $m$. The implementation of this algorithm is based on (\cite{yan2002number} pg. 123, \cite{rosen2011elementary} pg. 157). \\


\begin{algorithm}[H]
	\caption{Linear Congruence In One Variable}
	\DontPrintSemicolon
	\SetAlgoLined
	\KwIn{$a,b,x \in \mathbb{Z}$, $m \in \mathbb{N}$}
	\KwOut{$x$ general solution if any}
	
	\BlankLine
	
	Check solubility \\
	Let $g=GCD(a,m)$ \\
	\lIf{$g \nmid b$} { there is no solution modulo $m$. Stop. }
	\lIf{$g \mid b$} { there are $g$ incongruent solutions modulo $m$. Continue... }
	%\lIf{$g=1$} { There is a unique solution modulo $m$. Continue... }
	
	Use Extended Euclidean Algorithm to find $x_{ee}$ from $|a|x+|m|y=GCD(|a|,|m|)=g$ \\
	Set $x_{ee} = sign(a)x_{ee}$ \\
	A particular first initial solution is $x_0 = x_{ee}(b/g) \pmod{m}$ \\
	All initial solutions for $n = \lbrace0, ... ,g-1\rbrace$ are $x_n = n(m/g) + x_0 \pmod{m}$ \\
	For $r \in \mathbb{Z}$, any integer $x=mr+x_n$ is a solution \\
		
	\Return{$x$ general solution}
\end{algorithm}



\begin{thebibliography}{}
	\bibitem{yan2002number}
	Yan, Song Y.
	\newblock {\em Number Theory for Computing. - 2nd ed.}
	\newblock {Springer Science \& Business Media}, 2002.
	
	\bibitem{rosen2011elementary}
	Rosen, Kenneth H.
	\newblock {\em Elementary Number Theory and Its Applications. - 6th ed.}
	\newblock {Pearson Education London}, 2011.
	
	% Reference like: \cite{tattersall1999elementary} pg. 181
	%\bibitem{tattersall1999elementary}
	%Tattersall, James J.
	%\newblock {\em Elementary number theory in nine chapters. - 2nd ed.}
	%\newblock {Cambridge University Press}, 1999.	
	
	% Reference like: \cite{dickson1929introduction} pg. 10
	%\bibitem{dickson1929introduction}
	%Dickson, Leonard E.
	%\newblock {\em Introduction to the Theory of Numbers.}
	%\newblock {University of Chicago Press}, 1929.	
\end{thebibliography}



\end{document}






\begin{comment}
\begin{thebibliography}{}
\bibitem{yan2002number}
Yan, Song Y.
\newblock {\em Number theory for computing.}
\newblock {Springer Science \& Business Media}, 2002.
%\newblock
%\href{http://...}{Web source...}
\end{thebibliography}
%\section*{References}
%\newpage
% Source tutorial: https://www.youtube.com/watch?v=NRXvqthfDG0
% Use this link: https://scholar.google.com to find cites for references
% After using \cite command or adding new entry into the bib file. 
% Refer to (\cite{brown1999TonelliShanks} pg. 91)
% 1. Run pdflatex (F6)
% 2. Run BibTex (F11)
% 3. Run pdflatex (F6)
% 4. Run pdflatex (F6)
%\bibliographystyle{plain} % unsrt % 
%\bibliography{Bibliography}
% Display all bibliography cites even if there are no cites. Or just the cpecific ones
%\nocite{*} 
\end{comment}





\begin{comment}
\noindent
\begin{minipage}[t]{\minipageWidth\linewidth} % \linewidth, \textwidth, \hsize, \columnwidth
Column 1
\end{minipage}% PUT THIS % NOT TO HAVE AN EXTRA SPACE
\minipageSpaceCenter
\begin{minipage}[t]{\minipageWidth\linewidth}
Column 2
\end{minipage}% PUT THIS % NOT TO HAVE AN EXTRA SPACE
\end{comment}