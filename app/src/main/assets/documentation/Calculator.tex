\documentclass[12pt,oneside,a4paper]{article}
\usepackage[latin1]{inputenc}
\usepackage{amsmath}
\usepackage{amsfonts}
\usepackage{amssymb}
\usepackage{graphicx}
\usepackage{verbatim}
\usepackage{stmaryrd}
\usepackage{tikz}
% \usepackage{calc} % you're allowed to say \begin{minipage}{\linewidth-2ex}
%\usetikzlibrary{positioning,fit,shapes}
\usepackage[left=20mm, right=20mm, top=20mm, bottom=20mm]{geometry}
\usepackage{url} 
%\usepackage[hidelinks]{hyperref}  % For url links
\usepackage[colorlinks = true, %Colours links instead of ugly boxes
urlcolor = blue, %Colour for external hyperlinks blue, red, black
linkcolor = black, %Colour of internal links
citecolor = red %Colour of citations
]{hyperref}
\usepackage[ruled]{algorithm2e}
% [linesnumbered, ruled, vlined]
\usepackage{xcolor}
\definecolor{bgcolor}{HTML}{E3E1D7}

% Declare variables
\newcommand{\myScale}{0.345}
\newcommand{\myBorder}{0.0pt}
\newcommand{\minipageWidth}{0.485}
\newcommand{\minipageSpaceCenter}{\hspace{5mm}}

\title{Number Theory Algorithms}
\author{Ervin Gegprifti \\ [6pt]
	gegprifti.ervin@gmail.com}
\date{} 

\begin{document}
\pagecolor{bgcolor}
% \nopagecolor % Transparent if the rendering machine supports it.

\maketitle

\begin{abstract}
	This paper is the documentation for the Calculator module in \href{https://play.google.com/store/apps/details?id=com.gegprifti.android.numbertheoryalgorithms}{Number Theory Algorithms} mobile application.
\end{abstract}

\section*{Calculator operations}	
\subsection*{Addition: $a+b$}
\textbf{\emph{Description:}} Add $b$ to $a$. \\
\textbf{\emph{Input:}} $a,b$, where $a,b \in \mathbb{Z}$ \\ 
\textbf{\emph{Output:}} $a+b$

\subsection*{Subtraction: $a-b$}
\textbf{\emph{Description:}} Subtract $b$ from $a$. \\
\textbf{\emph{Input:}} $a,b$, where $a,b \in \mathbb{Z}$ \\
\textbf{\emph{Output:}} $a-b$

\subsection*{Multiplication: $a \times b$}
\textbf{\emph{Description:}} Multiply $a$ with $b$. \\
\textbf{\emph{Input:}} $a,b$, where $a,b \in \mathbb{Z}$ \\
\textbf{\emph{Output:}} $a \times b$

\subsection*{Division: $a/b$}
\textbf{\emph{Description:}} Divide $a$ with $b$. \\
\textbf{\emph{Input:}} $a,b$, where $a \in \mathbb{Z}$, $b \in \mathbb{Z}_{\ne 0}$ \\
\textbf{\emph{Output:}} quotient as $\lfloor a/b \rfloor$, remainder as $a-(\lfloor a/b \rfloor b)$

\subsection*{Power: $a^b$}
\textbf{\emph{Description:}} Raise $a$ to the power of $b$. \\
\textbf{\emph{Input:}} $a,b$, where $a \in \mathbb{Z}$, $b = \{0,\dots,2147483647\}$ \\
\textbf{\emph{Output:}} $a^b$

\subsection*{Root: $\sqrt[b]{a}$}
\textbf{\emph{Description:}} The $b$ root of $a$. \\
\textbf{\emph{Input:}} $a,b$, where $a \in \mathbb{Z}$, $b = \{1,\dots,2147483647\}$ \\
\textbf{\emph{Output:}} $\sqrt[b]{a}$

\subsection*{Greatest Common Divisor: $GCD(|a|, |b|)$}
\textbf{\emph{Description:}} The largest number that divides both $a$ and $b$ without leaving a remainder. \\
\textbf{\emph{Input:}} $a,b$, where $a,b \in \mathbb{Z}$ \\
\textbf{\emph{Output:}} $GCD(|a|, |b|)$

\subsection*{Lowest Common Multiple: $LCM(a,b)$}
\textbf{\emph{Description:}} The smallest integer that is evenly divisible by both $a$ and $b$. \\
\textbf{\emph{Input:}} $a,b$, where $a,b \in \mathbb{Z}$, not both $0$ \\
\textbf{\emph{Output:}} $LCM(a,b) = (ab)/GCD(a,b)$ since $(ab)=GCD(a,b)LCM(a,b)$

\subsection*{Modulo: $a \pmod b$}
\textbf{\emph{Description:}} The remainder when $a$ is divided by $b$. \\
\textbf{\emph{Input:}} $a,b$, where $a \in \mathbb{Z}$, $b \in \mathbb{Z}_{\geq 1}$ \\
\textbf{\emph{Output:}} $a \pmod b$, output is always a non-negative number

\subsection*{Modulo Inverse: $a^{-1} \pmod b$}
\textbf{\emph{Description:}} Modular inverse of $a \pmod b$ is $a^{-1}$. If $a \equiv c \pmod{b}$, then $aa^{-1} \equiv 1 \pmod{b}$. \\
\textbf{\emph{Input:}} $a$, where $a \in \mathbb{Z}$, $b \in \mathbb{Z}_{\geq 1}$ \\
\textbf{\emph{Output:}} $a^{-1} \pmod b$

\subsection*{Is probable prime:}
\textbf{\emph{Description:}} Check if a number is probable prime within a certain certainty. \\
\textbf{\emph{Input:}} $a$, where $a \in \mathbb{Z}$ with $a \ge 2$, $b = \{1, \dots, 2147483647\}$ \\
\textbf{\emph{Output:}} $1$ if $a$ is probably prime with probability $1-1/2^b$, $0$ if $a$ is definitely composite

\subsection*{Euler's phi-function: $\phi(a)$}
\textbf{Relatively prime definition.} The integers $d$ and $e$, with $d\neq0$ and $e\neq0$, are relatively prime if $d$ and $e$ have greatest common divisor $(d,e)=1$. Because $(25,42)=1$, then $25$ and $42$ are relatively prime. \\
\noindent
\textbf{Euler's phi-function $\phi(a)$ definition.} Let $a$ be a positive integer. The $\phi(a)$ is defined to be the number of positive integers not exceeding $a$ that are relatively prime to $a$. \\
\textbf{Example.} \\
\noindent
$\phi(1)=1$ because 
$\left\lbrace \begin{tabular}{ll}
	(1,1)=1 $\longrightarrow$ $counter=1$
\end{tabular} \right.$ \\ [6pt]
\noindent
$\phi(2)=1$ because 
$\left\lbrace \begin{tabular}{ll}
	(1,2)=1 $\longrightarrow$ $counter=1$ \\ [3pt]
	(2,2)=2
\end{tabular} \right.$ \\ [6pt]
\noindent
$\phi(3)=2$ because 
$\left\lbrace \begin{tabular}{ll}
	(1,3)=1 $\longrightarrow$ $counter=1$ \\ [3pt]
	(2,3)=1 $\longrightarrow$ $counter=2$ \\ [3pt]
	(3,3)=3
\end{tabular} \right.$ \\ [6pt]
\noindent
$\phi(4)=2$ because 
$\left\lbrace \begin{tabular}{ll}
	(1,4)=1 $\longrightarrow$ $counter=1$ \\ [3pt]
	(2,4)=2 \\ [3pt]
	(3,4)=1 $\longrightarrow$ $counter=2$ \\ [3pt]
	(4,4)=4
\end{tabular} \right.$ \\ [6pt]
\noindent
$\phi(5)=4$ because 
$\left\lbrace \begin{tabular}{ll}
	(1,5)=1 $\longrightarrow$ $counter=1$ \\ [3pt]
	(2,5)=1 $\longrightarrow$ $counter=2$ \\ [3pt]
	(3,5)=1 $\longrightarrow$ $counter=3$ \\ [3pt]
	(4,5)=1 $\longrightarrow$ $counter=4$ \\ [3pt]
	(5,5)=5
\end{tabular} \right.$

\subsection*{Factorial: $a!$}
\textbf{\emph{Description:}} Calculates the $a! = 1 \times 2 \times 3 \times \dots \times a$. \\
\textbf{\emph{Input:}} $a$, where $a \in \mathbb{Z}$ with $a > 0$ \\
\textbf{\emph{Output:}} $a!$

\subsection*{Next probable prime:}
\textbf{\emph{Description:}} The next probable prime to a number. \\
\textbf{\emph{Input:}} $a$, where $a \in \mathbb{Z}$ with $a \ge 2$ \\
\textbf{\emph{Output:}} next probable prime to $a$

\subsection*{Next twin prime to $a$:}
\textbf{\emph{Description:}} The next probable twin prime pair to $a$. \\
\textbf{\emph{Input:}} $a$, where $a \in \mathbb{Z}$ with $a > 2$ \\
\textbf{\emph{Output:}} next probable twin prime pair to $a$


\begin{thebibliography}{}
	\bibitem{javabiginteger}
	\newblock {"Class BigInteger."}
	\newblock \href{https://docs.oracle.com/javase/7/docs/api/java/math/BigInteger.html}{java.math.BigInteger}
\end{thebibliography}
%\section*{References}
%\newpage
% Source tutorial: https://www.youtube.com/watch?v=NRXvqthfDG0
% Use this link: https://scholar.google.com to find cites for references
% After using \cite command or adding new entry into the bib file. 
% Refer to (\cite{brown1999TonelliShanks} pg. 91)
% 1. Run pdflatex (F6)
% 2. Run BibTex (F11)
% 3. Run pdflatex (F6)
% 4. Run pdflatex (F6)
%\bibliographystyle{plain} % unsrt % 
%\bibliography{Bibliography}
% Display all bibliography cites even if there are no cites. Or just the cpecific ones
% \nocite{*} 

%CRC standard mathematical tables and formulae \\
%https://docs.oracle.com/javase/7/docs/api/java/math/BigInteger.html




\end{document}








\begin{comment}
\noindent
\begin{minipage}[t]{\minipageWidth\linewidth} % \linewidth, \textwidth, \hsize, \columnwidth
Column 1
\end{minipage}% PUT THIS % NOT TO HAVE AN EXTRA SPACE
\minipageSpaceCenter
\begin{minipage}[t]{\minipageWidth\linewidth}
Column 2
\end{minipage}% PUT THIS % NOT TO HAVE AN EXTRA SPACE
\end{comment}






\begin{comment}

$\boxbslash$
$\boxslash$


\begin{center}
\renewcommand\arraystretch{1.6}
\begin{tabular}{c|c}
$x$ 
&
$z$ \\
\end{tabular}
\end{center}

%\NewDocumentEnvironment


	% minipage reference: https://tex.stackexchange.com/tags/minipage/info
	\begin{minipage}[<pos>][<height>][<inner-pos>]{<width>} 
	<text>
	\end{minipage}
	where
	<pos> controls the vertical positioning of the box with respect to the text baseline (valid values: c, t, or b).
	<height> determines the height of the box.
	<width> determines the width of the box.
	<inner-pos> determines the position of <te