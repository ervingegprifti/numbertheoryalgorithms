\documentclass[12pt,oneside,a4paper]{article}
\usepackage[latin1]{inputenc}
\usepackage{amsmath}
\usepackage{amsfonts}
\usepackage{amssymb}
\usepackage{graphicx}
\usepackage{verbatim}
\usepackage{stmaryrd}
\usepackage{tikz}
% \usepackage{calc} % you're allowed to say \begin{minipage}{\linewidth-2ex}
%\usetikzlibrary{positioning,fit,shapes}
\usepackage[left=20mm, right=20mm, top=20mm, bottom=20mm]{geometry}
\usepackage{url} 
%\usepackage[hidelinks]{hyperref}  % For url links
\usepackage[colorlinks = true, %Colours links instead of ugly boxes
urlcolor = blue, %Colour for external hyperlinks blue, red, black
linkcolor = black, %Colour of internal links
citecolor = red %Colour of citations
]{hyperref}


%https://ftp.cc.uoc.gr/mirrors/CTAN/macros/latex/contrib/algorithm2e/doc/algorithm2e.pdf
\usepackage[linesnumbered,ruled,vlined]{algorithm2e}

% [linesnumbered, ruled, vlined]
\usepackage{xcolor}
\definecolor{bgcolor}{HTML}{E3E1D7}




% Declare variables
\newcommand{\myScale}{0.345}
\newcommand{\myBorder}{0.0pt}
\newcommand{\minipageWidth}{0.485}
\newcommand{\minipageSpaceCenter}{\hspace{5mm}}


\title{Number Theory Algorithms}
\author{Ervin Gegprifti \\ [6pt]
	gegprifti.ervin@gmail.com}
\date{} 

\begin{document}
\pagecolor{bgcolor}
% \nopagecolor % Transparent if the rendering machine supports it.

\maketitle

\begin{abstract}
	This paper is the documentation for the Linear Diophantine Equation In Two Variables module in \href{https://play.google.com/store/apps/details?id=com.gegprifti.android.numbertheoryalgorithms}{Number Theory Algorithms} mobile application.
\end{abstract}

\section*{Linear Diophantine Equation In Two Variables}	
The equation $ax+by=c$ where $a,b,x,y \in \mathbb{Z}$ with $a,b \neq 0$ has no integer solution if $GCD(a,b) \nmid c$ and many integer solutions if $GCD(a,b) \mid c$. The implementation of this algorithm is based on (\cite{rosen2011elementary} pg. 137, \cite{tattersall1999elementary} pg. 183). \\


\begin{algorithm}[H]
	\caption{Linear Diophantine Equation In Two Variables}
	\DontPrintSemicolon
	\SetAlgoLined
	\KwIn{$a,b,c,x,y \in \mathbb{Z}$ with $a,b \neq 0$}
	\KwOut{$x,y$ solutions if any}
	
	\BlankLine
	
	Set $g=GCD(a,b)$ \\
	\lIf{$g \nmid c$} { there is no integer solution. Stop. }
	\lIf{$g \mid b$} { there are infinitely many integer solutions. Continue.. }
	
	Use Extended Euclidean Algorithm to find $x_{ee}$ and $y_{ee}$ from $|a|x+|b|y=GCD(|a|,|b|)=g$ \\
	Set $x_{ee} = sign(a)x_{ee}$ and $y_{ee} = sign(b)y_{ee}$ \\
	A particular first initial solution is $x_0 = x_{ee}(c/g)$ and $y_0 = y_{ee}(c/g)$ \\
	For $r \in \mathbb{Z}$, any integer $x=x_0 + (b/g)r$ and $y=y_0 - (a/g)r$ is a solution \\
	
	\Return{solutions}
\end{algorithm}


\begin{thebibliography}{}
	\bibitem{rosen2011elementary}
	Rosen, Kenneth H.
	\newblock {\em Elementary Number Theory and Its Applications. - 6th ed.}
	\newblock {Pearson Education London}, 2011.
	
	\bibitem{tattersall1999elementary}
	Tattersall, James J.
	\newblock {\em Elementary number theory in nine chapters. - 2nd ed.}
	\newblock {Cambridge University Press}, 1999.	
\end{thebibliography}
%\section*{References}
%\newpage
% Source tutorial: https://www.youtube.com/watch?v=NRXvqthfDG0
% Use this link: https://scholar.google.com to find cites for references
% After using \cite command or adding new entry into the bib file. 
% Refer to (\cite{brown1999TonelliShanks} pg. 91)
% 1. Run pdflatex (F6)
% 2. Run BibTex (F11)
% 3. Run pdflatex (F6)
% 4. Run pdflatex (F6)
%\bibliographystyle{plain} % unsrt % 
%\bibliography{Bibliography}
% Display all bibliography cites even if there are no cites. Or just the cpecific ones
%\nocite{*} 


\end{document}