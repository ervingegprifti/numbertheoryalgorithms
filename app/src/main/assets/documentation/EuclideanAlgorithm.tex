\documentclass[12pt,oneside,a4paper]{article}
\usepackage[latin1]{inputenc}
\usepackage{amsmath}
\usepackage{amsfonts}
\usepackage{amssymb}
\usepackage{graphicx}
\usepackage{verbatim}
\usepackage{stmaryrd}
\usepackage{tikz}
% \usepackage{calc} % you're allowed to say \begin{minipage}{\linewidth-2ex}
%\usetikzlibrary{positioning,fit,shapes}
\usepackage[left=20mm, right=20mm, top=20mm, bottom=20mm]{geometry}
\usepackage{url} 
%\usepackage[hidelinks]{hyperref}  % For url links
\usepackage[colorlinks = true, %Colours links instead of ugly boxes
urlcolor = blue, %Colour for external hyperlinks blue, red, black
linkcolor = black, %Colour of internal links
citecolor = red %Colour of citations
]{hyperref}
\usepackage[ruled]{algorithm2e}
% [linesnumbered, ruled, vlined]
\usepackage{xcolor}
\definecolor{bgcolor}{HTML}{E3E1D7}


% Declare variables
\newcommand{\myScale}{0.345}
\newcommand{\myBorder}{0.0pt}
\newcommand{\minipageWidth}{0.485}
\newcommand{\minipageSpaceCenter}{\hspace{5mm}}


\title{Number Theory Algorithms}
\author{Ervin Gegprifti \\ [6pt]
	gegprifti.ervin@gmail.com}
\date{} 

\begin{document}
\pagecolor{bgcolor}
% \nopagecolor % Transparent if the rendering machine supports it.

\maketitle

\begin{abstract}
	This paper is the documentation for the Euclidean Algorithm module in \href{https://play.google.com/store/apps/details?id=com.gegprifti.android.numbertheoryalgorithms}{Number Theory Algorithms} mobile application.
\end{abstract}

\section*{Euclidean Algorithm}	
The Euclidean Algorithm is used to compute the greatest common divisor (GCD) of two numbers $a$ and $b$. The (GCD) is the largest number that divides both $a$ and $b$ without leaving a remainder. The implementation of this algorithm is based on (\cite{yan2002number} pg. 40). \\

\begin{algorithm}[H]
\DontPrintSemicolon
\SetAlgoLined
\KwIn{$a,b \in \mathbb{Z}$}
\KwOut{The greatest common divisor (GCD) of  $a$ and $b$}
\BlankLine
\lIf{$a<0$}{$a = |a|$}
\lIf{$b<0$}{$b = |b|$}
\lIf{$a=b$}{\Return{$a$, since $a|a$ and $a|b$}}
\lIf{$a\neq 0$ {\bf and} $b=0$}{\Return{$a$}}
\lIf{$a = 0$ {\bf and} $b \neq 0$}{\Return{$b$}}
\lIf{$a=0$ {\bf and} $b=0$}{\Return{$0$}}
\lIf{$b | a$}{\Return{$b$}}
\BlankLine
$r_{n-2}:=a$ \;
$r_{n-1}:=b$ \;
$q_{n-1}:=$ quotient of $r_{n-2}/r_{n-1}$ \;
$r_n:=$ remainder of $r_{n-2}/r_{n-1}$ \;
\BlankLine
\While{$r_n > 0$}
{
  $r_{n-2}:=r_{n-1}$ \;
  $r_{n-1}:=r_n$ \;
  $q_{n-1}:=$ quotient of $r_{n-2}/r_{n-1}$ \;
  $r_n:=$ remainder of $r_{n-2}/r_{n-1}$ \;
}
\BlankLine
\Return{$r_{n-1}$}
\caption{Euclidean Algorithm}
\end{algorithm}




\begin{thebibliography}{}
	
	\bibitem{yan2002number}
	Yan, Song Y.
	\newblock {\em Number theory for computing.}
	\newblock {Springer Science \& Business Media}, 2002.
	%\newblock
	%\href{http://...}{Web source...}
\end{thebibliography}
%\section*{References}
%\newpage
% Source tutorial: https://www.youtube.com/watch?v=NRXvqthfDG0
% Use this link: https://scholar.google.com to find cites for references
% After using \cite command or adding new entry into the bib file. 
% Refer to (\cite{brown1999TonelliShanks} pg. 91)
% 1. Run pdflatex (F6)
% 2. Run BibTex (F11)
% 3. Run pdflatex (F6)
% 4. Run pdflatex (F6)
%\bibliographystyle{plain} % unsrt % 
%\bibliography{Bibliography}
% Display all bibliography cites even if there are no cites. Or just the cpecific ones
%\nocite{*} 



\end{document}



















\begin{comment}
@book{yan2002number,
title={Number theory for computing},
author={Yan, Song Y},
year={2002},
publisher={Springer Science \& Business Media}
}
\end{comment}



\begin{comment}
\noindent
\begin{minipage}[t]{\minipageWidth\linewidth} % \linewidth, \textwidth, \hsize, \columnwidth
Column 1
\end{minipage}% PUT THIS % NOT TO HAVE AN EXTRA SPACE
\minipageSpaceCenter
\begin{minipage}[t]{\minipageWidth\linewidth}
Column 2
\end{minipage}% PUT THIS % NOT TO HAVE AN EXTRA SPACE
\end{comment}






\begin{comment}

$\boxbslash$
$\boxslash$


\begin{center}
\renewcommand\arraystretch{1.6}
\begin